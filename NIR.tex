\documentclass[a4paper]{article}

%Пакеты для математических символов:

\usepackage{amsmath} % американское математическое сообщество.
\usepackage{amssymb} % миллион разных значков и готический, ажурный шрифты.
\usepackage{amscd} % диаграммы, графики.
\usepackage{amsthm} % окружения теорем, определений и тд.
\usepackage{physics} % основные физические символы
\usepackage{tikz-feynhand} % Ферми диограммы
%\usepackage{latexsym} % треугольники и пьяная стрелка.

%пакеты для шрифтов:
%\usepackage{euscript} % прописной шрифт с завитушками.
\usepackage{MnSymbol} % Значеки доказательства
\usepackage{verbatim} % улучшенный шрифт "пишущей машинки".
\usepackage{array} % более удобные таблицы.
%\usepackage{multirow} % мультистолбцы в таблицах.
%\usepackage{longtable} % таблицы на несколько страниц.
%\usepackage{latexsym}
\usepackage{collectbox} % Добавляет коробочки, можно складывать туда текст)

\usepackage[backend=biber, style=gost-numeric]{biblatex} % Литература по госту


\usepackage{hyperref} % Ссылки как внешние так и внутренние
\hypersetup{
    colorlinks=true,
    linkcolor=black,
    filecolor=magenta,      
    urlcolor=cyan,
    pdftitle={Overleaf Example},
    pdfpagemode=FullScreen,
    }
    
%Пакеты для оформления:
\RequirePackage[center, medium]{titlesec}% Стиль секций и заголовков
%\usepackage[x11names]{xcolor} % 317 новых цветов для текста.
\usepackage{float} % Позволяет использовать H, h! для локации фигур
%\usepackage{multicol} % набор текста в несколько колонн.
\usepackage{graphicx} % расширенные возможности вставки стандартных картинок.
\usepackage{subcaption} % возможность вставлять картинки в строчку
%\usepackage{caption} % возможность подавить нумерацию у caption.
\usepackage{wrapfig} % вставка картинок и таблиц, обтекаемых текстом.
\usepackage{cancel} % значки для сокращения дробей, упрощения, стремления.
\usepackage{misccorr} % в заголовках появляется точка, но при ссылке на них ее нет.
\usepackage{indentfirst} % отступ у первой строки раздела
%\usepackage{showkeys} % показывает label формул над их номером.
%\usepackage{fancyhdr} % удобное создание верхних и нижних колонтитулов.
%\usepackage{titlesec} % еще одно создание верхних и нижних колонтитулов
\usepackage{hyperref} % Ссылки как внешние так и внутренние
\hypersetup{
    colorlinks=true,
    linkcolor=black,
    filecolor=magenta,      
    urlcolor=cyan,
    pdftitle={Overleaf Example},
    pdfpagemode=FullScreen,
    }
\usepackage{xcolor} %Позволяет перекрасить все страници
\definecolor{mycolor}{RGB}{244,228,215} %Цвет перекраски


%Пакеты шрифтов, кодировок. НЕ МЕНЯТЬ РАСПОЛОЖЕНИЕ.
\usepackage[utf8]{inputenc} % кодировка символов.
%\usepackage{mathtext} % позволяет использовать русские буквы в формулах. НЕСОВМЕСТИМО С tempora.
\usepackage[T1, T2A]{fontenc} % кодировка шрифта.
\usepackage[english, russian]{babel} % доступные языки.

%tikz:
\usepackage{tikz} % tikz
\usetikzlibrary{decorations.text} % позволяет делать текст вдоль кривой.
\usetikzlibrary{external} % позволяет кэшировать рисунки tikz.


\usepackage{pgfplotstable}

%Отступы и поля:
%размеры страницы А4 11.7x8.3in
\textwidth=7.3in % ширина текста
\textheight=10in % высота текста
\oddsidemargin=-0.5in % левый отступ(базовый 1дюйм + значение)
\topmargin=-0.5in % отступ сверху до колонтитула(базовый 1дюйм + значение)


%Сокращения
%Скобочки
\newcommand{\inrad}[1]{\left( #1 \right)}
\newcommand{\inner}[1]{\left( #1 \right)}
\newcommand{\infig}[1]{\left\{ #1 \right\}}
\newcommand{\insqr}[1]{\left[ #1 \right]}
\newcommand{\ave}[1]{\left\langle #1 \right\rangle}



%% Красивые <= и >=
\renewcommand{\geq}{\geqslant}
\renewcommand{\leq}{\leqslant}

%%Значек выполнятся
\newcommand{\per}{\hookrightarrow}

%%Векторная алгебра
\newcommand{\rot}{\text{rot}}
\renewcommand{\div}{\text{div}}
\renewcommand{\grad}{\text{grad}}

%% Более привычные греческие буквы
\renewcommand{\phi}{\varphi}
\renewcommand{\epsilon}{\varepsilon}
\newcommand{\eps}{\varepsilon}
\newcommand{\com}{\mathbb{C}}
\newcommand{\re}{\mathbb{R}}
\newcommand{\nat}{\mathbb{N}}
\newcommand{\stp}{$\filledmedtriangleleft$}
\newcommand{\enp}{$\filledmedsquare$}

%%Тензорный анализ ОТО теория поля
\newcommand{\Li}[1]{\mathfrak{L}_{#1}}
\newcommand{\crist}[3]{\cfrac{1}{2} \inner{g_{#1#2,#3} + g_{#1#3,#2} - g_{#2#3,#1}}}
\newcommand{\piv}[2]{\cfrac{\partial #1}{\partial #2}}


\newcommand{\redd}[1]{\textcolor{red}{#1}}%Красный текст
\newcommand{\newdot}{\textbullet \hspace{0.5em}}%абзац начинающийся с точки


\makeatletter
\newcommand{\sqbox}{%
    \collectbox{%
        \@tempdima=\dimexpr\width-\totalheight\relax
        \ifdim\@tempdima<\z@
            \fbox{\hbox{\hspace{-.5\@tempdima}\BOXCONTENT\hspace{-.5\@tempdima}}}%
        \else
            \ht\collectedbox=\dimexpr\ht\collectedbox+.5\@tempdima\relax
            \dp\collectedbox=\dimexpr\dp\collectedbox+.5\@tempdima\relax
            \fbox{\BOXCONTENT}%
        \fi
    }%
}
\makeatother
\newcommand{\mergelines}[2]{
\begin{tabular}{llp{.5\textwidth}}
#1 \\ #2
\end{tabular}
}
\newcommand\tab[1][0.51cm]{\hspace*{#1}}
\newcommand\difh[2]{\frac{\partial #1}{\partial #2}}
\newcommand{\messageforpeople}[1]{HSE Faculty of Physics \ \ HSE Faculty of Physics HSE Faculty of Physics \ \ HSE Faculty of Physics HSE Faculty of Physics \ \ HSE Faculty of Physics HSE Faculty of Physics \ \ HSE Faculty of Physics HSE Faculty of Physics \ \ HSE Faculty of Physics HSE Faculty of Physics \ \ HSE Faculty of Physics HSE Faculty of Physics \ \ HSE Faculty of Physics HSE Faculty of Physics \ \ HSE Faculty of Physics }

\newcounter{customsection}
\newcommand{\mysection}[1]{
    \refstepcounter{customsection}
    \addcontentsline{toc}{section}{Appendix \arabic{customsection}: #1} % Добавляет запись в содержание
    \noindent\textbf{Appendix \arabic{customsection}: #1}
}



\numberwithin{equation}{section}


\begin{document}

\newpage
\tableofcontents
\newpagestyle{main}{
\setfootrule{0.4pt}
\setfoot{}{\thepage}{\sectiontitle }}
\pagestyle{main}


\section{Введение}

\subsection{Мотивация}

Предпосылки открытия очарованного бариона $\Lambda_c$ появились в 1975 году, когда в результате наблюдения аномалии в распаде $e^+ e^- \to e^+ + \mu^- + E_{miss}$ (см. \textbf{\cite{PhysRevLett1975}}) было высказано предположение о существовании заряженного лёгкого очарованного бариона. Открытие на достаточном уровне значимости произошло более чем 10 лет спустя на коллайдере SPEAR (см. \textbf{\cite{Avery1988}}) по распаду $\Lambda_c \to p K^- \pi^+$. 

$\Lambda_c$, будучи самым лёгким из очарованных барионов, распадается исключительно посредством слабого взаимодействия, что позволяет изолировать и исследовать вклад этого взаимодействия в барионных системах.
В частности, канал $\Lambda_c \rightarrow \Lambda l \nu_l$, где $l = e, \mu$, а распад с продуктом $l = \tau$ подавлен в силу закона сохранения 4-импульса:

\begin{equation*}
    m_{\Lambda_c} = 2.28646\,\text{GeV} < 2.89261\,\text{GeV} = 1.77693\,\text{GeV} + 1.11568\,\text{GeV} = m_{\tau} + m_{\Lambda}.
\end{equation*}

Бранчинговые отношения для полулептонных распадов $\Lambda_c \rightarrow \Lambda l \nu_l$, где $l = e, \mu$, были измерены в нескольких работах. Для канала $\Lambda_c \rightarrow \Lambda e \nu_e$ измеренное бранчинговое отношение составляет $B(\Lambda_c \rightarrow \Lambda e \nu_e) = 3.56 \pm 0.13\%$, как указано в статье \cite{CLEO2023}. Для канала $\Lambda_c \rightarrow \Lambda \mu \nu_{\mu}$ измеренное бранчинговое отношение равно $B(\Lambda_c \rightarrow \Lambda \mu \nu_{\mu}) = 3.48 \pm 0.17\%$ согласно \cite{CLEO2023}.

Полулептонные распады $\Lambda_c$ являются удобным и относительно простым случаем для исследования переходов тяжелого кварка в лёгкий, что позволяет точнее проверять предсказания теоретических моделей, таких как эффективная теория тяжёлых кварков (HQET) и квантовая хромодинамика на решётке (LQCD). Проверка этих моделей с помощью экспериментов может не только подтвердить их верность, но и выявить отклонения от стандартной модели, что потенциально указывает на существование новой физики, включая новые взаимодействия или экзотические частицы.

\subsection{Отличие от работы CLEO}

Измерение форм-фактора $\Lambda_c \rightarrow \Lambda l \nu_l$ важно для проверки результатов предыдущего эксперимента \textbf{\cite{CLEO2023}}, в котором был измерен форм-фактор $\Lambda_c \rightarrow \Lambda e \nu_e$. Важно сравнить методологические и экспериментальные аспекты текущего исследования с работой команды CLEO.

Прежде всего, команда CLEO сделала предположение о том, что спин бариона $\Lambda$ равномерно распределён. Это предположение оказывает влияние на значение спиральности, которое напрямую входит в уравнение для форм-фактора. В данной работе предлагается более точное измерение распределения направлений спина, основанное на анализе распада в канале $\Lambda_c^+ \rightarrow \Lambda \pi^+$. Этот подход позволяет уменьшить систематические ошибки и повысить точность вычислений.

Второе важное отличие заключается в использовании независимого источника данных. В то время как команда CLEO использовала данные, собранные с детектора "CLEO" на Корнельском электронном накопительном кольце (Cornell Electron Storage Ring), в настоящей работе анализ проводился на детекторе "Belle", установленном на ускорителе "KEK". Это не только обеспечивает независимую проверку результатов, но и позволяет уточнить их с учётом различий в экспериментальных установках.

Наконец, команда CLEO не проводила анализа полулептонного распада $\Lambda_c \rightarrow \Lambda \mu \nu_\mu$, что является существенным упущением. В данном исследовании этот канал был тщательно изучен, что позволяет расширить понимание полулептонных распадов и улучшить тесты на универсальность лептонов.

Таким образом, данная работа вносит вклад в дальнейшее изучение свойств бариона $\Lambda_c$ и уточнение результатов, полученных в предыдущих исследованиях.

\subsection{Модели и теоретические предсказания}

Как уже было сказано выше, на данный момент существуют численные методы вычисления форм-факторов, основанные на различных приближениях или моделях. Все они дают различные результаты (см. таблицу \ref{tb:chd}). Поскольку сильное взаимодействие сложно поддаётся теоретическим расчётам из-за отсутствия малого параметра, результаты могут быть проверены только экспериментально.

\begin{table}[H]
    \centering
    \caption{Форм-факторы полулептонных распадов $\Lambda_c \to \Lambda$ для $q^2 = 0$.}
    \begin{tabular}{|c|c|c|c|c|c|c|}
    \hline
    Form Factor & $\mathfrak{F}_1^V(0)$ & $\mathfrak{F}_2^V(0)$ & $\mathfrak{F}_3^V(0)$ & $\mathfrak{F}_1^A(0)$ & $\mathfrak{F}_2^A(0)$ & $\mathfrak{F}_3^A(0)$ \\
    \hline
    \textbf{\cite{QCD2021}} & 0.687(138) & 0.486(117) & 0.164(80) & 0.539(101) & -0.388(100) & -0.359(283) \\
    \hline
    \textbf{\cite{BagModel1989}} & 0.35 & 0.09 & 0.25 & 0.61 & -0.04 & -0.11 \\
    \hline
    \textbf{\cite{RQM2016}} & 1.14 & 0.072 & 0.252 & 0.517 & -0.697 & -0.471 \\
    \hline
    \textbf{\cite{QSR2009}} & 0.665 & 0.285 & --- & 0.665 & -0.285 & --- \\
    \hline
    \textbf{\cite{LFCQM2018}} & 0.468 & 0.222 & --- & 0.407 & -0.035 & --- \\
    \hline
    \end{tabular}
    \label{tb:chd}
\end{table}


\section{Каналов тагирования}
\label{taging}
\subsection{Тагирование $\Lambda_c$}

Для восстановления распадов $\Lambda_c$-барионов и определения импульса недетектируемого нейтрино применяется тагирование по заряду, аромату и барионному числу. Будем предполагать, что $\Lambda_c$ образуется из $\bar{c}$-кварка и подхваченных из вакуума недостающих кварков. В таком случае будем называть систему центромасс $c$-кварка $X_c$, то есть неизвестная очарованная частица, которая фактически может быть не одной, а несколькими частицами сразу.

\begin{figure}[h!]
    \centering
    \begin{tikzpicture}
        \begin{feynhand}
            \vertex [particle] (e) at (-2,1) {$e^-$};
            \vertex [particle] (ae) at (-2,-1) {$e^+$};
            
            \vertex [particle] (photon) at (0,0);
            \vertex (w1) at (1, 0);
            
            \vertex [particle] (c) at (3,1) {$c$};
            \vertex [particle] (anti_c) at (3,-1) {$\bar{c}$};
        
            \propag [fermion] (e) to (photon);
            \propag [anti fermion] (ae) to (photon);
            \propag [photon] (photon) to  [edge label=$\gamma*$]  (w1);
            \propag [fermion] (w1) to (c);
            \propag [anti fermion] (w1) to (anti_c);
        \end{feynhand}
    \end{tikzpicture}
\end{figure}

Для того чтобы определить состав $X_c$, необходимо, чтобы соблюдались законы 
сохранения барионного числа, аромата, заряда, а также 4-импульса. Такая 
технология называется тагированием \textcolor{red}{ссылка на работу}. В 
результате получим, что в $X_c$ будет входить хотя бы один барион и кварки 
$u c d$, а также любые пары $q \bar{q}$. В итоге возможны следующие варианты 
$X_c$. Также важно понимать, что чем больше частиц содержит $X_c$, тем менее 
вероятно событие с такой комбинацией, так как новые частицы требуют 
дополнительных кварковых пар, создание которых требует больше энергии. 
Кроме того, при добавлении новых частиц время работы программы увеличивается 
экспоненциально, так как сложность алгоритма $\mathcal{O}(\prod_n N_n)$ 
(где $N_n$ количестов задетектированных частиц типа $n$ в событии).

В работе рассматриваются $X_c \to \Lambda^{tag}_c; \Lambda^{tag}_c \pi^- \pi^-; \Lambda^{tag}_c \pi^+ \pi^- \pi^+ \pi^-; D^0 p; D^+ p \pi^-; D*^0 p; D*^+ p \pi^- $, 
чтобы отличать $\Lambda_c$ котрую тагируемую от тагирующей (той что является продуктом $X_c$), вторую обозначаим как $\Lambda^{tag}_c$.
Каналы распада прочих частиц будем импользовать заведомо изветные самые эффективные каналы, согласно \cite{PDGTablesBar} для барионов и \cite{PDGTablesMes} для мезонов.
\begin{figure}[h]
    \centering
    \begin{tabular}{c|c}
        Particle & Channels \\ \hline
        $D^0$ & $K^- \pi^+; K^- \pi^+ \pi^+ \pi^-; K^- K^+; K^0_s \pi^+ \pi^+; K^0_s \pi^0; K^+ K^- K_s^0$ \\
        $D^+$ & $K^- \pi^+ \pi^+; K^0_s; K^0_s \pi^+ \pi^+ \pi^-; K^+ K^- \pi^+$ \\
        $\Lambda^{tag}_c$ & $pK^-\pi^+; \Lambda^0 \pi^+; \Lambda^0 \pi^+ \pi^0; p K_s^0 \pi^0$ \\
        $D*^+$ & $D^0 \pi^0; D^0 \gamma$ \\
        $D*^+$ & $D^+ \pi^0; D^0 \pi^+$ \\
        $\pi^0$ & $\gamma \gamma$ \\
        $K_s^0$ & $\pi^+ \pi^-$
    \end{tabular}
    \label{fig:part_channels}
\end{figure}

\subsection{Критерии отбора}

В данном разделе изложены критерии отбора, принятые на основании работы \cite*{BelleDetector2002} и описанном в \ref{mes_mthods}. 
Имея на набор треков и их параметров, надо их классифицировать по типу частици оставившей этот трек. 

\newdot Фотоны классифицированы, но используемые при реконструкции событий, наложим дополнительное ограничение $E_\gamma > 50 \text{ MeV}$, поскольку фотоны с меньшей энергией трудно отличимы тормозных или индуцированных в сичтеме токов, 
что может привести к ошибочной интерпретации их как сигнальных фотонов.

\newdot Идентификация частиц по (PID):

Как уже известно для треков формируется значение правдоподобия $L(p,a)$ 
и в поледствии PID значение $\mathfrak{L}_{p_1/p_2}(a)$, поэтому на треки котрые хотим 
идентифицировать как частицу $p$ наложим следующие ограничения:

\begin{figure}[h]
    \centering
    \begin{tabular}{c|c}
        Гипотеза & Критекрий \\ \hline
        $p \& \bar p$ & $\mathfrak{L}_{p/K} < 0.6; \mathfrak{L}_{p/\pi} > 0.6$ \\
        $K^\pm$   & $\mathfrak{L}_{p/K} < 0.4; \mathfrak{L}_{K/\pi} > 0.6$ \\
        $\pi^{\pm}$ & все заряженные треки, не прошедшие идентификацию по вышеуказанным критериям \\
    \end{tabular}
\end{figure}

\newdot $K_s^0$-мезоны реконструируются по распаду $K_s^0 \to \pi^+ \pi^-$ из кандидатов, отобранных с помощью стандартного инструмента V0finder и собранных в таблице MdstVee2. Критерии отбора следующие:

$$
\left| M_{K_s^0} - M^{real}_{K_s^0} \right| < 30 \text{ MeV}; \ \rho_{K_s^0} > 1 \text{ мм}; \ z_{K_s^0} > 1 \text{ см}; \ \cos \theta_{K_s^0} > 0.99
$$

где $M^{real}_{K_s^0} = 497.611 \text{ MeV}$, $M_{K_s^0}$ — инвариантная масса пионов ($\pi^+ \pi^-$), собранных в $K_s^0$-мезон, $z_{K_s^0}$ и $\rho_{K_s^0}$ — цилиндрические координаты реконструированной вершины распада $K_s^0$-мезона в лабораторной системе отсчёта, а $\cos \theta_{K_s^0}$ — азимутальный угол между импульсом $K_s^0$ и направлением на его вершину распада.

\newdot $\pi^0$ -мезоны восстанавливались в распаде на два фотона, которые в
свою очередь реконструировались по кластерам энерговыделения в ECL.
Критерии отбора:
$$
\abs{M_{\pi^0} - M_{\pi^0}^{real}} < 15 MeV 
$$
После отбора стандартно были установлены погрешности для импульсов фотонов и выполнены фиты в вершину и массу.

\newdot Отбор $D$ -мезонов:

\begin{figure}[h]
    \centering
    \begin{tabular}{c|c}
        $D^0$ & $\abs{M_{D^0} - M^{real}_{D^0}} < 15 MeV$ \\
        $D^\pm$ & $\abs{M_{D^\pm} - M^{real}_{D^\pm}} < 15 MeV$ \\
        $D*^{\pm}$ & $\abs{M_{D*^\pm} - M^{real}_{D*^\pm}} < 3 MeV$ \\
        $D*^0$ & $\abs{M_{D*^0} - M^{real}_{D*^0}} < 3 MeV$ \\
    \end{tabular}
\end{figure}

Где $M^{real}_{D^\pm} = 1864.83 MeV; M^{real}_{D^0} = 1869.65 MeV; 
M^{real}_{D*^\pm} = 2010.26;  M^{real}_{D*^0} = 2006.85 MeV$. 
Так де на $D*$-мезоны накладываем ограничение:

\begin{equation}
        \abs{M_{D*} - M^d_{D}} < 15 MeV
\end{equation}

Где $ M^d_{D}$ --- масса $D$-мезона используемого для кобинации соответствующего. 
$D^*$. Так как ошибка собранного $D*$ мезона карелирует с собранным 
предврительно $D$-мезоном.



\mysection{Метод вычисления формфактора}

$\Lambda_c$ барион состит из $ucd$ кварков, в ходе распараспада 
$\Lambda_c \rightarrow \Lambda l \nu_l$ проискходит переход $c\to s$ 
посредством испускания $W^+$ бозона тоесть правиьно будет записвть $c\to s W^+$,
$W^+$ распадается на $W^+ \to l^+ \nu_l$, в итоге оставшиеся кварки 
$uds$ формируются в $\Lambda$ барион. Таким образом получим следующую 
феймановскую диаграмму.

\begin{figure}[H]
    \centering
    \begin{tikzpicture}
        \begin{feynhand}
            \vertex [particle] (i1) at (-3,4) {$u$};
            \vertex [particle] (i2) at (-3,3.5) {$d$};
            \vertex [particle] (i3) at (-3,3) {$c$};
            \vertex [particle] (f1) at (3,4) {$u$};
            \vertex [particle] (f2) at (3,3.5) {$d$};
            \vertex [particle] (f3) at (3,3) {$s$};
            \vertex (w1) at (0,3);
            \vertex (w2) at (0,2.5);
            \vertex (w3) at (0,2);
            \vertex (w4) at (1.5,1);
            \vertex [particle] (e) at (3,1.5) {$l^{-}$};
            \vertex [particle] (an) at (3,0.5) {$\bar{\nu}_{l}$};
            \propag [fermion] (i1) to (w1);
            \propag [fermion] (i2) to (w2);
            \propag [fermion] (i3) to (w3);
            \propag [fermion] (w1) to (f1);
            \propag [fermion] (w2) to (f2);
            \propag [fermion] (w3) to (f3);
            \propag [charged boson] (w3) to [edge label=$W^{-}$] (w4);
            \propag [fermion] (w4) to (e);
            \propag [anti fermion] (w4) to (an);        
        \end{feynhand}
    \end{tikzpicture}
\end{figure}

Переход $\Lambda_c \to \Lambda$ индуцирется слабым током $j_\mu$, 
котрый можно разложить по аксиальной и векторной части: 
$j_\mu = j_\mu^A + j_\mu^V$.
Обозначим волновые функции частиц
$B_{\Lambda_c} \inner{p_{\Lambda_c}, M_{\Lambda_c}} 
\to B_{\Lambda} \inner{p_{\Lambda}, M_{\Lambda}} 
+ l\inner{p_l, m_l} + \nu_l \inner{p_\nu, m = 0}$. 
Форм факторы выражаются как:
\begin{equation}
    \bra{B_{\Lambda_c} \inner{p_{\Lambda_c}, M_{\Lambda_c}}}
    j_\nu^V
    \ket{B_{\Lambda} \inner{p_{\Lambda}, M_{\Lambda}}} = 
    u_2^\dag \inner{\mathfrak F^V_1 \inner{q^2} \gamma_\nu + 
    \cfrac{\mathfrak F^V_2}{M_{\Lambda_c}} \inner{q^2} \sigma_{\mu\nu} q^\nu + 
    \cfrac{\mathfrak F^V_3}{M_{\Lambda_c}} \inner{q^2}q_\mu} u_1 
\end{equation}

\begin{equation}
    \bra{B_{\Lambda_c} \inner{p_{\Lambda_c}, M_{\Lambda_c}}}
    j_\nu^A
    \ket{B_{\Lambda} \inner{p_{\Lambda}, M_{\Lambda}}} = 
    u_2^\dag \inner{\mathfrak F^A_1 \inner{q^2} \gamma_\nu + 
    \cfrac{\mathfrak F^A_2}{M_{\Lambda_c}} \inner{q^2} \sigma_{\mu\nu} q^\nu + 
    \cfrac{\mathfrak F^V_3}{M_{\Lambda_c}} \inner{q^2}q_\mu} \gamma_5 u_1 
\end{equation}

Где $\gamma_\mu$ - матрци Диррака, $q_\mu$ - 4-импульс $W^+$ бозона,
$\sigma_{\mu\nu} = \cfrac{1}{2} \inner{\gamma_\mu \gamma_\nu - \gamma_\nu \gamma_\mu}$.

\redd{Дописать вывод связи форм фактора и спиральности}


\section{Литература}

\begin{thebibliography}{99}

    \bibitem{Avery1988} 
    Avery P., Blanco R., Liu K., et al. Observation of the Charmed Baryon $\Lambda^+_c$ at SPEAR // Phys. Rev. Lett. 1988. V. 50. P. 747-750. DOI: 10.1103/PhysRevLett.50.747.
    
    \bibitem{PhysRevLett1975} 
    Perl M. L., Abrams G. S., Boyarski A. M., et al. Evidence for Anomalous Lepton Production in $e^+e^-$ Annihilation // Phys. Rev. Lett. 1975. V. 35. P. 1129-1132. DOI: 10.1103/PhysRevLett.35.1129.
    
    \bibitem{CLEO2022} 
    Eisenstein B. I., Alexander J. P., Berkelman K. Study of the Semileptonic Decay $\Lambda_c \rightarrow \Lambda e \nu_e$ // Physical Review D. 2022. V. 105. P. 012007. DOI: 10.1103/PhysRevD.105.012007.
    
    \bibitem{CLEO2023} 
    Dobbs S., Metreveli Z., Seth K. K. Study of $\Lambda_c^+ \rightarrow \Lambda \mu^+ \nu_{\mu}$ and test of lepton flavor universality with $\Lambda_c^+ \rightarrow \Lambda l^+ \nu_l$ decays // Physical Review D. 2023. V. 106. P. 032005. DOI: 10.1103/PhysRevD.106.032005.
    
    \bibitem{BagModel1989} 
    Perez-Marcial R., Huerta R., Garcia A., Avila-Aoki M. Predictions for semileptonic decays of charm baryons. 2. Nonrelativistic and MIT bag quark models // Phys. Rev. D. 1989. V. 40. P. 2955. DOI: 10.1103/PhysRevD.40.2955.
    
    \bibitem{RQM2016} 
    Faustov R. N., Galkin V. O. Semileptonic decays of $\Lambda_c$ baryons in the relativistic quark model // Eur. Phys. J. C. 2016. V. 76. P. 628. DOI: 10.1140/epjc/s10052-016-4492-z.
    
    \bibitem{QSR2009} 
    Liu Y. L., Huang M. Q., Wang D. W. Improved analysis on the semi-leptonic decay $\Lambda_c \to \Lambda l \nu$ from QCD light-cone sum rules // Phys. Rev. D. 2009. V. 80. P. 074011. DOI: 10.1103/PhysRevD.80.074011.
    
    \bibitem{LFCQM2018} 
    Zhao Z. X. Weak decays of heavy baryons in the light-front approach // Chin. Phys. C. 2018. V. 42. P. 093101. DOI: 10.1088/1674-1137/42/9/093101.
    
    \bibitem{LFCQM2021} 
    Geng C. Q., Liu C. W., Tsai T. H. Semileptonic weak decays of antitriplet charmed baryons in the light-front formalism // Phys. Rev. D. 2021. V. 103. P. 054018. DOI: 10.1103/PhysRevD.103.054018.
    
    \bibitem{CQM2016} 
    Gutsche T., Ivanov M. A., Korner J. G., Lyubovitskij V. E., Santorelli P. Semileptonic decays $\Lambda_c \to \Lambda \ell \nu$ in the covariant quark model // Phys. Rev. D. 2016. V. 93. P. 034008. DOI: 10.1103/PhysRevD.93.034008.    
    
    \bibitem{QCD2021} 
    Bahtiyar H., Can K. U., Oka M., Takahashi T. T. $\Lambda_c \to \Lambda$ Form Factors in Lattice QCD // Phys. Rev. D. 2021. V. 102. P. 114505. DOI: 10.1103/PhysRevD.102.114505.    

    \bibitem{PDGTablesBar}
    Navas S., et al. (Particle Data Group). Review of Particle Physics // Phys. Rev. D. 2024. V. 110. $№$ 3. P. 030001.(2024)
    
    \bibitem{PDGTablesMes}
    Navas S. et al. (Particle Data Group). Review of Particle Physics // Phys. Rev. D. 2024. V. 110. № 3. P. 030001. (2024)

    \bibitem{BelleDetector2002}
    Abashian A. et al. The Belle Detector // Nuclear Instruments and Methods in Physics Research A. 2002. V. 479. P. 117–232.
    
    \bibitem{Krohn2021}
    Krohn J.-F., Urquijo P., Abudinén F., et al. Global Decay Chain Vertex Fitting at B-Factories // Nuclear Instruments and Methods in Physics Research A. 2021. V. 988. P. 164891.

    \bibitem{Richman}
    Richman, J. D. An Experimenter’s Guide to the Helicity Formalism // J. D. Richman // CALT-68-1148.


\end{thebibliography}



\end{document}
