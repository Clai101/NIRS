\mysection{Алгоритм фита в массу потеряной частици}

\begin{figure}[h!]
    \centering
    \begin{tikzpicture}
        \begin{feynhand}
            \vertex [particle] (e) at (-3,0) {$e^-$};
            \vertex [particle] (ae) at (3,0) {$e^+$};
            
            \vertex [dot] (w1) at (0, 0) {};
                        
            \vertex [particle] (p1) at (1.964,-0.377) {$p_1$};
            \vertex [particle] (p2) at (-1.350,1.476) {$p_2$};
            \vertex [particle] (p3) at (1.582,1.222) {$p_3$};
            \vertex [particle] (lam_c) at (-1.667,-1.108) {$p_{taging}$};
        
            \propag [fermion] (e) to (w1);
            \propag [fermion] (ae) to (w1);
            \propag [fermion] (w1) to (p1);
            \propag [fermion] (w1) to (p2);
            \propag [fermion] (w1) to (p3);
            \propag [fermion] (w1) to (lam_c);
        \end{feynhand}
    \end{tikzpicture}
    \caption{Схема распада.}
    \label{fit_tag}
\end{figure}

Дано: $p_1, p_2, p_3$ --- это 4-импульсы продуктов распада (тагирующих выбранную частицу), представленных на рис.~\ref{fit_tag}. 
Также известны матрицы ковариаций компонент 3-импульса $\Xi_1, \Xi_2, \Xi_3$ для соответствующих частиц. 
$p_{\text{beam}}$ — это 4-импульс системы ($p_{\text{beam}}$), а $M_{\text{rec}}$ --- это масса недостающей (тагируемой) частицы.

Для поиска оптимального решения используется метод множителей Лагранжа. Поскольку мы минимизируем изменения импульсов с учётом их ошибок, применяем следующую функцию:

\begin{equation}
    \chi^2 = \sum_n \inner{p_{n}}_i  \inner{\Xi_n^{-1}}_{ij}  \inner{p_{n}}_j
\end{equation}

Функция Лагранжа с наложением ограничения имеет вид:

\begin{equation}
    M_{\text{rec}}^2 - (p_{\text{beam}} - \sum_n p_n)_\mu (p_{\text{beam}} - \sum_n p_n)^\mu = 0
\end{equation}

Полная функция Лагранжа с множителем Лагранжа $\lambda$ записывается следующим образом:

\begin{equation}
    \mathfrak{L}(p_n, \lambda) = 
    sum_n \inner{p_{n}}_i  \inner{\Xi_n^{-1}}_{ij}  \inner{p_{n}}_j + 
    \lambda \left( M_{\text{rec}}^2 - (p_{\text{beam}} - \sum_n p_n)_\mu (p_{\text{beam}} - \sum_n p_n)^\mu \right)
\end{equation}

Для минимизации используется метод Ньютона-Рафсона. На каждом шаге вычисляются градиент (первая производная) и гессиан (матрица вторых производных) функции Лагранжа:

\begin{equation}
    \nabla \mathfrak{L}(x_n) = c_n, \Delta \otimes \Delta \mathfrak{L}(x_n)  = \hat{A}_n
\end{equation}

где $x_n$ — это вектор параметров на $n$-м шаге. Затем на каждом шаге решается система уравнений для обновления параметров:

\begin{equation}
    \hat{A}_n \, \delta x_{n+1} = -c_n
\end{equation}

После чего параметры обновляются:

\begin{equation}
    x_{n+1} = x_n + \delta x_{n+1}.
\end{equation}



