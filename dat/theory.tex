\mysection{Метод вычисления формфактора}

$\Lambda_c$ барион состит из $ucd$ кварков, в ходе распараспада 
$\Lambda_c \rightarrow \Lambda l \nu_l$ проискходит переход $c\to s$ 
посредством испускания $W^+$ бозона тоесть правиьно будет записвть $c\to s W^+$,
$W^+$ распадается на $W^+ \to l^+ \nu_l$, в итоге оставшиеся кварки 
$uds$ формируются в $\Lambda$ барион. Таким образом получим следующую 
феймановскую диаграмму.

\begin{figure}[H]
    \centering
    \begin{tikzpicture}
        \begin{feynhand}
            \vertex [particle] (i1) at (-3,4) {$u$};
            \vertex [particle] (i2) at (-3,3.5) {$d$};
            \vertex [particle] (i3) at (-3,3) {$c$};
            \vertex [particle] (f1) at (3,4) {$u$};
            \vertex [particle] (f2) at (3,3.5) {$d$};
            \vertex [particle] (f3) at (3,3) {$s$};
            \vertex (w1) at (0,3);
            \vertex (w2) at (0,2.5);
            \vertex (w3) at (0,2);
            \vertex (w4) at (1.5,1);
            \vertex [particle] (e) at (3,1.5) {$l^{-}$};
            \vertex [particle] (an) at (3,0.5) {$\bar{\nu}_{l}$};
            \propag [fermion] (i1) to (w1);
            \propag [fermion] (i2) to (w2);
            \propag [fermion] (i3) to (w3);
            \propag [fermion] (w1) to (f1);
            \propag [fermion] (w2) to (f2);
            \propag [fermion] (w3) to (f3);
            \propag [charged boson] (w3) to [edge label=$W^{-}$] (w4);
            \propag [fermion] (w4) to (e);
            \propag [anti fermion] (w4) to (an);        
        \end{feynhand}
    \end{tikzpicture}
\end{figure}

Переход $\Lambda_c \to \Lambda$ индуцирется слабым током $j_\mu$, 
котрый можно разложить по аксиальной и векторной части: 
$j_\mu = j_\mu^A + j_\mu^V$.
Обозначим волновые функции частиц
$B_{\Lambda_c} \inner{p_{\Lambda_c}, M_{\Lambda_c}} 
\to B_{\Lambda} \inner{p_{\Lambda}, M_{\Lambda}} 
+ l\inner{p_l, m_l} + \nu_l \inner{p_\nu, m = 0}$. 
Форм факторы выражаются как:
\begin{equation}
    \bra{B_{\Lambda_c} \inner{p_{\Lambda_c}, M_{\Lambda_c}}}
    j_\nu^V
    \ket{B_{\Lambda} \inner{p_{\Lambda}, M_{\Lambda}}} = 
    u_2^\dag \inner{\mathfrak F^V_1 \inner{q^2} \gamma_\nu + 
    \cfrac{\mathfrak F^V_2}{M_{\Lambda_c}} \inner{q^2} \sigma_{\mu\nu} q^\nu + 
    \cfrac{\mathfrak F^V_3}{M_{\Lambda_c}} \inner{q^2}q_\mu} u_1 
\end{equation}

\begin{equation}
    \bra{B_{\Lambda_c} \inner{p_{\Lambda_c}, M_{\Lambda_c}}}
    j_\nu^A
    \ket{B_{\Lambda} \inner{p_{\Lambda}, M_{\Lambda}}} = 
    u_2^\dag \inner{\mathfrak F^A_1 \inner{q^2} \gamma_\nu + 
    \cfrac{\mathfrak F^A_2}{M_{\Lambda_c}} \inner{q^2} \sigma_{\mu\nu} q^\nu + 
    \cfrac{\mathfrak F^V_3}{M_{\Lambda_c}} \inner{q^2}q_\mu} \gamma_5 u_1 
\end{equation}

Где $\gamma_\mu$ - матрци Диррака, $q_\mu$ - 4-импульс $W^+$ бозона,
$\sigma_{\mu\nu} = \cfrac{1}{2} \inner{\gamma_\mu \gamma_\nu - \gamma_\nu \gamma_\mu}$.



\redd{Дописать вывод связи форм фактора и спиральности}


