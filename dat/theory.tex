\mysection{Метод вычисления формфактора}
 
$\Lambda_c$-барион состоит из $ucd$ кварков. В ходе распада 
$\Lambda_c \rightarrow \Lambda l \nu_l$ происходит переход $c \to s$ 
посредством испускания $W^+$-бозона, то есть правильно будет записать $c \to s W^+$.
$W^+$ распадается на $W^+ \to l^+ \nu_l$, в итоге оставшиеся кварки 
$uds$ формируются в $\Lambda$-барион. Таким образом, получим следующую 
фейнмановскую диаграмму.
 
\begin{figure}[H]
    \centering
    \begin{tikzpicture}
        \begin{feynhand}
            \vertex [particle] (i1) at (-3,4) {$u$};
            \vertex [particle] (i2) at (-3,3.5) {$d$};
            \vertex [particle] (i3) at (-3,3) {$c$};
            \vertex [particle] (f1) at (3,4) {$u$};
            \vertex [particle] (f2) at (3,3.5) {$d$};
            \vertex [particle] (f3) at (3,3) {$s$};
            \vertex (w1) at (0,3);
            \vertex (w2) at (0,2.5);
            \vertex (w3) at (0,2);
            \vertex (w4) at (1.5,1);
            \vertex [particle] (e) at (3,1.5) {$l^{-}$};
            \vertex [particle] (an) at (3,0.5) {$\bar{\nu}_{l}$};
            \propag [fermion] (i1) to (w1);
            \propag [fermion] (i2) to (w2);
            \propag [fermion] (i3) to (w3);
            \propag [fermion] (w1) to (f1);
            \propag [fermion] (w2) to (f2);
            \propag [fermion] (w3) to (f3);
            \propag [charged boson] (w3) to [edge label=$W^{-}$] (w4);
            \propag [fermion] (w4) to (e);
            \propag [anti fermion] (w4) to (an);        
        \end{feynhand}
    \end{tikzpicture}
\end{figure}
 
Переход $\Lambda_c \to \Lambda$ индуцируется слабым током $j_\mu$, 
который можно разложить по аксиальной и векторной частям: 
$j_\mu = j_\mu^A + j_\mu^V$.
Обозначим волновые функции частиц
$B_{\Lambda_c} \inner{p_{\Lambda_c}, M_{\Lambda_c}} 
\to B_{\Lambda} \inner{p_{\Lambda}, M_{\Lambda}} 
+ l\inner{p_l, m_l} + \nu_l \inner{p_\nu, m = 0}$. 
Формфакторы выражаются как:
\begin{eqnarray}
    \bra{B_{\Lambda_c} \inner{p_{\Lambda_c}, M_{\Lambda_c}}}
    j_\nu^V
    \ket{B_{\Lambda} \inner{p_{\Lambda}, M_{\Lambda}}} = 
    u_\Lambda^\dag \inner{\mathfrak F^V_1 \inner{q^2} \gamma_\nu + 
    \cfrac{\mathfrak F^V_2}{M_{\Lambda_c}} \inner{q^2} \sigma_{\mu\nu} q^\nu + 
    \cfrac{\mathfrak F^V_3}{M_{\Lambda_c}} \inner{q^2} q_\mu} u_{\Lambda_c} \\
    \bra{B_{\Lambda_c} \inner{p_{\Lambda_c}, M_{\Lambda_c}}}
    j_\nu^A
    \ket{B_{\Lambda} \inner{p_{\Lambda}, M_{\Lambda}}} = 
    u_\Lambda^\dag \inner{\mathfrak F^A_1 \inner{q^2} \gamma_\nu + 
    \cfrac{\mathfrak F^A_2}{M_{\Lambda_c}} \inner{q^2} \sigma_{\mu\nu} q^\nu + 
    \cfrac{\mathfrak F^A_3}{M_{\Lambda_c}} \inner{q^2} q_\mu} \gamma_5 u_{\Lambda_c} 
    \label{eq:def_form}
\end{eqnarray}
 
Где $\gamma_\mu$ --- матрицы Дирака, $q_\mu$ --- 4-импульс $W^+$-бозона,
$\sigma_{\mu\nu} = \cfrac{1}{2} \inner{\gamma_\mu \gamma_\nu - \gamma_\nu \gamma_\mu}$.
 
Из уравнения Дирака мы знаем, что биспинор нерелятивистской частицы выглядит как:
 
\begin{gather}
    u = 
    \begin{pmatrix}
        \phi_\pm \\
        \chi_\pm
    \end{pmatrix}
    =
    \begin{pmatrix}
        \chi_\pm \\
        \cfrac{\mp p}{E - M^2} \chi_\pm
    \end{pmatrix};
    \phi_\pm = \cfrac{\mp p}{E - M^2} \chi_\pm.
\end{gather}
 
Где $p$, $E$, $M$ --- импульс, энергия и масса частицы в произвольной системе отсчета, $\chi$, $\phi$ --- связанные друг с другом спиноры частицы. 
Но можно рассматривать процесс в системе отсчета $\Lambda_c$, 
тогда биспинор $\Lambda_c$ примет вид:
 
\begin{gather}
    u_{\Lambda_c} = \sqrt{2 M_{\Lambda_c}}
    \begin{pmatrix}
        \chi_\pm \\
        0
    \end{pmatrix},
\end{gather}
 
а биспинор $\Lambda$: 
 
\begin{gather}
    u_{\Lambda} = \sqrt{E_{\Lambda} + M_{\Lambda}}
    \begin{pmatrix}
        \chi_\pm \\
        \cfrac{\mp \abs{p_{\Lambda_c}}}{E - M^2} \chi_\pm
    \end{pmatrix}.
\end{gather}
 
Где $\chi$ --- соответствующие спиноры частиц:
\begin{gather}
    \chi_+ = 
    \begin{pmatrix}
        1 \\
        0
    \end{pmatrix}; \quad
    \chi_- = 
    \begin{pmatrix}
        0 \\
        1
    \end{pmatrix}.
\end{gather}
 
А 4-импульс $W$-бозона в той же системе отсчета примет вид:
 
\begin{gather}
    q^\mu = 
    \begin{pmatrix}
        q_0 \\ 0 \\ 0 \\ -p
    \end{pmatrix}.
\end{gather}
 
Спиральность определяется как:
 
\begin{equation}
    H_{\lambda_\Lambda \lambda_w} = H^V_{\lambda_\Lambda \lambda_w} 
    + H^A_{\lambda_\Lambda \lambda_w} 
\end{equation}
 
\begin{equation}
    H^{V,A}_{\lambda_\Lambda \lambda_w} = 
    \bra{B_{\Lambda_c} \inner{p_{\Lambda_c}, M_{\Lambda_c}}}
    j_\nu^{V,A}
    \ket{B_{\Lambda} \inner{p_{\Lambda}, M_{\Lambda}}} 
    \eps^\nu \inner{\lambda_w}
    \label{eq:helisity}
\end{equation}
 
И в стойкой системе отсчета 4-вектор поляризации, первое число показывает 
проекцию спина на импульс, а второе --- полный спин:
 
\begin{gather}
    \eps^\nu\inner{0, 0} = \cfrac{1}{q} 
    \begin{pmatrix}
        q \\ 0 \\ 0 \\ -p
    \end{pmatrix},
    \quad
    \eps^\nu\inner{\pm 1, 1} = \cfrac{1}{\sqrt{2}} 
    \begin{pmatrix}
        0 \\ \pm 1 \\ -i \\ 0
    \end{pmatrix},
    \quad
    \eps^\nu\inner{0, 1} = \cfrac{1}{q} 
    \begin{pmatrix}
        p \\ 0 \\ 0 \\ -q
    \end{pmatrix}
    \label{eq:pol}
\end{gather}
 
Из законов сохранения импульса и энергии можно также вывести, что:
 
\begin{equation}
    q_0 = \frac{1}{2 M_{\Lambda_c}} \inner{ M_{\Lambda_c}^2 - M_{\Lambda}^2 + q^2 },
\end{equation}
 
\begin{equation}
p = \abs{\vec{p}_2} = \frac{1}{2 M_{\Lambda_c}} \sqrt{\inner{ M_{\Lambda_c} + M_{\Lambda} }^2 - q^2} \sqrt{\inner{ M_{\Lambda_c} - M_{\Lambda} }^2 - q^2},
\end{equation}
 
Простая подстановка \ref{eq:pol} и \ref{eq:def_form} в \ref{eq:helisity} даст нам:
 
\begin{equation}
H^{V}_{\frac{1}{2}t} = \frac{\sqrt{Q_+}}{\sqrt{q^2}} \inrad{ F_1^V \inrad{ M_{\Lambda_c} - M_{\Lambda} } + F_3^V \frac{q^2}{M_{\Lambda_c}} },
\end{equation}
\begin{equation}
H^{V}_{\frac{1}{2}1} = \sqrt{2Q_-} \inrad{ - F_1^V - F_2^V \frac{M_{\Lambda_c} + M_{\Lambda}}{M_{\Lambda_c}} },
\end{equation}
\begin{equation}
H^{V}_{\frac{1}{2}0} = \frac{\sqrt{Q_-}}{\sqrt{q^2}} \inrad{ F_1^V \inrad{ M_{\Lambda_c} + M_{\Lambda} } + F_2^V \frac{q^2}{M_{\Lambda_c}} },
\end{equation}
\begin{equation}
H^{A}_{\frac{1}{2}t} = \frac{\sqrt{Q_-}}{\sqrt{q^2}} \inrad{ - F_1^A \inrad{ M_{\Lambda_c} + M_{\Lambda} } + F_3^A \frac{q^2}{M_{\Lambda_c}} },
\end{equation}
\begin{equation}
H^{A}_{\frac{1}{2}1} = \sqrt{2Q_+} \inrad{ F_1^A - F_2^A \frac{M_{\Lambda_c} - M_{\Lambda}}{M_{\Lambda_c}} },
\end{equation}
\begin{equation}
H^{A}_{\frac{1}{2}0} = \frac{\sqrt{Q_+}}{\sqrt{q^2}} \inrad{ - F_1^A \inrad{ M_{\Lambda_c} - M_{\Lambda} } + F_2^A \frac{q^2}{M_{\Lambda_c}} },
\end{equation}
 
при этом выполняются следующие соотношения чётности:
\begin{equation}
H^{V}_{-\lambda_\Lambda,-\lambda_W} = H^{V}_{\lambda_\Lambda,\lambda_W}, \quad
H^{A}_{-\lambda_\Lambda,-\lambda_W} = - H^{A}_{\lambda_\Lambda,\lambda_W}.
\end{equation}
 
Для удобства записи было введено обозначение $t$ как проекция 0 в случае полного спина 0.