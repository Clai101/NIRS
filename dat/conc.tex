\section{Вывод}

В данной работе было проведено исследование полулептонных распадов бариона $\Lambda_c$, 
с целью измерения его продольной поляризации в каналах $\Lambda_c \to \Lambda e \nu_e; \Lambda \pi$ 
и форм-факторов в канеле $\Lambda_c \to \Lambda_c e \nu_e$. 
Экспериментальные данные, собранные с помощью детектора Belle на коллайдере KEK, 
Почти на полной статистике эксперимента Belle было отобрано около
$20$ тысяч распадов $\Lambda_c$-барионов в процессе $e^+ e^- \to \Lambda_c X_c$ путём 
восстановления частиц в наборе $X_c$ и использования законов сохранения
энергии и импульса, а также обученной на $MC$ данный моделью.

Среди тагированных $\Lambda_c$ были отобраны события с распадами $\Lambda_c \to \Lambda \pi$. 
Для них получена формула углового распределения, с помощью аппроксимации которой была измерена
продольная поляризация тагированных $\Lambda_c$-барионов: $ P_L = 0.64\pm 0.16$,
что соответствует ожиданиям теоретических моделей и подтверждает их состоятельность.

Среди тагированных $\Lambda_c$ были отобраны события с распадами $\Lambda_c \to \Lambda l \nu_l$.
Для них получена упрощенная формула углового распределения, с помощью аппроксимации которой была измерена
продольная поляризация тагированных $\Lambda_c$-барионов:
\begin{eqnarray}
    \lambda_c \to \Lambda e \nu_e: \ R = -0.52 \pm 0.13,\ M_{pole} = 1.82 ± 0.19 GeV
\end{eqnarray}
Задающих форм-факторы. 

Полученные на данный момте разультаты согласуются, с полученными CLEO результатами, использующих 
назависимые данные и иное приближение.