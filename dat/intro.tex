\section{Введение}

\subsection{Мотивация}

Предпосылки открытия очарованного бариона $\Lambda_c$ появились в 1975 году, когда в результате наблюдения аномалии в распаде $e^+ e^- \to e^+ + \mu^- + E_{miss}$ (см. \cite{PhysRevLett1975}) было высказано предположение о существовании заряженного легкого очарованного бариона. Открытие на достаточном уровне значимости произошло более чем 10 лет спустя на коллайдере SPEAR (см. \cite{Avery1988}) по распаду $\Lambda_c \to p K^- \pi^+$. $\Lambda_c$, будучи самым легким из очарованных барионов, распадается исключительно посредством слабого взаимодействия, что позволяет изолировать и исследовать вклад этого взаимодействия в барионных системах.

В частности, канал $\Lambda_c \rightarrow \Lambda l \nu_l$, где $l = e, \mu$, а распад с продуктом $l = \tau$ подавлен в силу закона сохранения 4-импульса:

\begin{equation*}
    m_{\Lambda_c} = 2.28646\,\text{GeV} < 2.89261\,\text{GeV} = 1.77693\,\text{GeV} + 1.11568\,\text{GeV} = m_{\tau} + m_{\Lambda}.
\end{equation*}

Бранчинговые отношения для полулептонных распадов $\Lambda_c \rightarrow \Lambda l \nu_l$, где $l = e, \mu$, были измерены в нескольких работах. Для канала $\Lambda_c \rightarrow \Lambda e \nu_e$ измеренное бранчинговое отношение составляет $B(\Lambda_c \rightarrow \Lambda e \nu_e) = 3.56 \pm 0.13\%$, как указано в статье \cite{CLEO2022}. Для канала $\Lambda_c \rightarrow \Lambda \mu \nu_{\mu}$ измеренное бранчинговое отношение равно $B(\Lambda_c \rightarrow \Lambda \mu \nu_{\mu}) = 3.48 \pm 0.17\%$ согласно \cite{CLEO2023}.

Полулептонные распады $\Lambda_c$ являются удобным и относительно простым случаем для исследования переходов тяжелого кварка в легкий, что позволяет точнее проверять предсказания теоретических моделей, таких как эффективная теория тяжелых кварков (HQET) и квантовая хромодинамика на решетке (LQCD). Проверка этих моделей с помощью экспериментов может не только подтвердить их верность, но и выявить отклонения от стандартной модели, что потенциально указывает на существование новой физики, включая новые взаимодействия или экзотические частицы.

\textcolor{red}{Возможно, стоит написать про внедрение $FEI$, так как это может быть важно для дальнейшего развития области. Нейронные сети всегда звучат круто для людей, которые в них не разбираются.}

\subsection{Отличие от работы CLEO}

Измерение форм-фактора $\Lambda_c \rightarrow \Lambda l \nu_l$ важно для проверки результатов предыдущего эксперимента \cite{CLEO2023}, в котором был измерен форм-фактор $\Lambda_c \rightarrow \Lambda e \nu_e$. Важно сравнить методологические и экспериментальные аспекты текущего исследования с работой команды CLEO.

Прежде всего, команда CLEO сделала предположение о том, что спин бариона $\Lambda$ равномерно распределен. Это предположение оказывает влияние на значение спиральности, которое напрямую входит в уравнение для форм-фактора. В данной работе предлагается более точное измерение распределения направлений спина, основанное на анализе распада в канале $\Lambda_c^+ \rightarrow \Lambda \pi^+$. Этот подход позволяет уменьшить систематические ошибки и повысить точность вычислений.

Второе важное отличие заключается в использовании независимого источника данных. В то время как команда CLEO использовала данные, собранные с детектора "CLEO" на Корнельском электронном накопительном кольце (Cornell Electron Storage Ring), в настоящей работе анализ проводился на детекторе "Belle", установленном на ускорителе "KEK". Это не только обеспечивает независимую проверку результатов, но и позволяет уточнить их с учётом различий в экспериментальных установках.

Наконец, команда CLEO не проводила анализа полулептонного распада $\Lambda_c \rightarrow \Lambda \mu \nu_\mu$, что является существенным упущением. В данном исследовании этот канал был тщательно изучен, что позволяет расширить понимание полулептонных распадов и улучшить тесты на универсальность лептонов.

Таким образом, данная работа вносит вклад в дальнейшее изучение свойств бариона $\Lambda_c$ и уточнение результатов, полученных в предыдущих исследованиях.

\subsection{Модели и теоретические предсказания}



\begin{table}[H]
    \centering
    \caption{Формфакторы полулептонных распадов $\Lambda_c \to \Lambda$ для $q^2 = 0$ (q --- 4 bvgekmc $W*$).}
    \begin{tabular}{|c|c|c|c|c|c|c|}
    \hline
    Form Factor & $\mathfrak{F}_1^V(0)$ & $\mathfrak{F}_2^V(0)$ & $\mathfrak{F}_3^V(0)$ & $\mathfrak{F}_1^A(0)$ & $\mathfrak{F}_2^A(0)$ & $\mathfrak{F}_3^A(0)$ \\
    \hline
    \cite{QCD2021} & 0.687(138) & 0.486(117) & 0.164(80) & 0.539(101) & -0.388(100) & -0.359(283) \\
    \hline
    \cite{BagModel1989} & 0.35 & 0.09 & 0.25 & 0.61 & -0.04 & -0.11 \\
    \hline
    \cite{RQM2016} & 1.14 & 0.072 & 0.252 & 0.517 & -0.697 & -0.471 \\
    \hline
    \cite{QSR2009} & 0.665 & 0.285 & --- & 0.665 & -0.285 & --- \\
    \hline
    \cite{LFCQM2018} & 0.468 & 0.222 & --- & 0.407 & -0.035 & --- \\
    \hline
    \end{tabular}
\end{table}

    
    
