\section{Поиск $\Lambda_c$}

На основе выбранных нами каналов $X_c$ писанных в предыдущем разделе 
обираем события в котрых выполняется условие 

\begin{equation}
    \abs{p_{e^+} + p_{e^-} - p_{X_c}}^2 \leq 3 GeV
\end{equation}

Так как в идельных условиях эта влечина должна быть равна квадрату $M^{real}_{\Lambda_c} = 2226.46 MeV$, 
таким образом мы откинем множетво событий котрые затагировали 
возбужденные состояния $\Lambda_c$ или поряли трек.

В отобранных событиях мы будем собирать $\Lambda_c$ барионы. 
По каналам $\Lambda_c^+ \to \Lambda \pi^+; \Lambda \nu_e e^+; \Lambda \nu_\mu \mu^+$.


\newdot Для отбора $e^\pm$ требем $p_{e^\pm} \geq 0.6 GeV$, чтобы долетел до SVD 
детектора где он в ходе $e^- \to \gamma e^- \to 2e^- e^+$ распадается электронн-фотонным ливнем, 
что позволяет его олично индетефицировать, поэтому критерий на $L(e) > 0.1$ не такой строгий.

\newdot Аналогично отбора $\mu^\pm$ требем $p_{\mu^\pm} \geq 0.6 GeV$, чтобы долетел до KLM, 
где идентификация мюнов еще лучше поэтому требуем $L(\mu) > 0.01$ 

\newdot Для отбора $\Lambda \to p \pi$ требуем 

$$
\abs{ M_{\Lambda} - M^{real}_{\Lambda} } < 30 \text{ MeV}; 
\ \rho_{\Lambda} > 1 \text{ mm}; \ z_{\Lambda} > 1 \text{ cm}; 
\ \cos \theta_{\Lambda} > 0.99.
$$


\newdot Комбинируем $\Lambda_c$ с массовым окном $50 MeV$.

\newdot Независимо от работы \cite{BelleDetector2002}, среди моножетва $\infig{d_n}$, 
идентифицированных как кандидаты на дочерние продукты распада $\xi$ частици, 
можно использовать знание о том, что импульсы продуктов распада должны исходить 
из вершины распада. Это позволяет откорректировать измеренные импульсы с учётом 
погрешностей, чтобы они соответствовали данной гипотезе (в дальнейшем это будет 
называться "фит в вершину"). Аналогично, на основании инвариантной массы, 
известной для $\xi$, можно корректировать величины импульсов дочерних частиц 
так, чтобы $M_{p} = \sqrt{\sum_n \inner{p_n}_\gamma \inner{p_n}^\gamma}$ 
совпадала с $M^{real}_{\xi}$, где $p_n$ --- 4-импульс сответствующий $d_n$ 
из $\infig{d_n}$. Этот метод будет называться "фит в массу". Были использованы 
алгоритмы для фита в вершину и в массу принятые в коллаборации KEK, и описанные 
в \cite{Krohn2021}.

По итогу делаем фиты в вершину, а после в массу для всех собранных частиц это 
$\Lambda_c, D^{\pm}, D^0, D^{*\pm}, D^{*0}, \Lambda, K_s^0$.

\newdot Для импульс $X_c$ фитируем так чтобы $\abs{p_{e^+} + p_{e^-} - p_{X_c}}^2 = M^2_{\Lambda_c}$, 
подробно алгоритм описан в Appendix \ref{fit_rec_mass}. 

 







