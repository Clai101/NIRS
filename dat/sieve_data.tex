\section{Сбор данных}

Для восстановления распадов $\Lambda_c$-барионов и определения импульса недетектируемого нейтрино применяется тагирование по заряду, аромату и барионному числу. Будем предполагать, что $\Lambda_c$ образуется из $\bar{c}$-кварка и подхваченных из вакуума недостающих кварков. В таком случае будем называть систему центромасс $c$-кварка $X_c$, то есть неизвестная очарованная частица, которая фактически может быть не одной, а несколькими частицами сразу.

\begin{figure}[h!]
    \centering
    \begin{tikzpicture}
        \begin{feynhand}
            \vertex [particle] (e) at (-2,1) {$e^-$};
            \vertex [particle] (ae) at (-2,-1) {$e^+$};
            
            \vertex [particle] (photon) at (0,0);
            \vertex (w1) at (1, 0);
            
            \vertex [particle] (c) at (3,1) {$c$};
            \vertex [particle] (anti_c) at (3,-1) {$\bar{c}$};
        
            \propag [fermion] (e) to (photon);
            \propag [anti fermion] (ae) to (photon);
            \propag [photon] (photon) to  [edge label=$\gamma*$]  (w1);
            \propag [fermion] (w1) to (c);
            \propag [anti fermion] (w1) to (anti_c);
        \end{feynhand}
    \end{tikzpicture}
\end{figure}

Для того чтобы определить состав $X_c$, необходимо, чтобы соблюдались законы сохранения барионного числа, аромата, заряда, а также 4-импульса. Такая технология называется тагированием \textcolor{red}{ссылка на работу}. В результате получим, что в $X_c$ будет входить хотя бы один барион и кварки $u c d$, а также любые пары $q \bar{q}$. В итоге возможны следующие варианты $X_c$. Также важно понимать, что чем больше частиц содержит $X_c$, тем менее вероятно событие с такой комбинацией, так как новые частицы требуют дополнительных кварковых пар, создание которых требует больше энергии. Кроме того, при добавлении новых частиц время работы программы увеличивается экспоненциально, так как сложность алгоритма $\mathfrak{O}(C^n)$ \textcolor{red}{в аппендиксе покажу, что это так, и добавлю ссылку}.

Я рассмотрю $X_c \to \Lambda^{tag}_c; \Lambda^{tag}_c \pi^- \pi^-; \Lambda^{tag}_c \pi^+ \pi^- \pi^+ \pi^-; D^0 p; D^+ p \pi^-; D*^0 p; D*^+ p \pi^- $, чтобы отличать $\Lambda_c$ котрую мы такгируем от той что является продуктом $X_c$, вторую обозначаю как $\Lambda^{tag}_c$.
Каналы распада прочих частиц, так как для этой отальных частиц нам уже извесины бранчинговые соотношения, будут использованы самые эффективные из них согласно \cite{PDGTablesBar} для барионов и \cite{PDGTablesMes} для мезонов.
\begin{figure}[h]
    \centering
    \begin{tabular}{c|c}
        Particle & Channels \\ \hline
        $D^0$ & $K^- \pi^+; K^- \pi^+ \pi^+ \pi^-; K^- K^+; K^0_s \pi^+ \pi^+; K^0_s \pi^0; K^+ K^- K_s^0$ \\
        $D^+$ & $K^- \pi^+ \pi^+; K^0_s; K^0_s \pi^+ \pi^+ \pi^-; K^+ K^- \pi^+$ \\
        $\Lambda^{tag}_c$ & $pK^-\pi^+; \Lambda^0 \pi^+; \Lambda^0 \pi^+ \pi^0; p K_s^0 \pi^0$ \\
    \end{tabular}
    \label{fig:channels}
\end{figure}


