\section{Сбор данных}

\subsection{Тагирование $\Lambda_c$}

Для восстановления распадов $\Lambda_c$-барионов и определения импульса недетектируемого нейтрино применяется тагирование по заряду, аромату и барионному числу. Будем предполагать, что $\Lambda_c$ образуется из $\bar{c}$-кварка и подхваченных из вакуума недостающих кварков. В таком случае будем называть систему центромасс $c$-кварка $X_c$, то есть неизвестная очарованная частица, которая фактически может быть не одной, а несколькими частицами сразу.

\begin{figure}[h!]
    \centering
    \begin{tikzpicture}
        \begin{feynhand}
            \vertex [particle] (e) at (-2,1) {$e^-$};
            \vertex [particle] (ae) at (-2,-1) {$e^+$};
            
            \vertex [particle] (photon) at (0,0);
            \vertex (w1) at (1, 0);
            
            \vertex [particle] (c) at (3,1) {$c$};
            \vertex [particle] (anti_c) at (3,-1) {$\bar{c}$};
        
            \propag [fermion] (e) to (photon);
            \propag [anti fermion] (ae) to (photon);
            \propag [photon] (photon) to  [edge label=$\gamma*$]  (w1);
            \propag [fermion] (w1) to (c);
            \propag [anti fermion] (w1) to (anti_c);
        \end{feynhand}
    \end{tikzpicture}
\end{figure}

Для того чтобы определить состав $X_c$, необходимо, чтобы соблюдались законы 
сохранения барионного числа, аромата, заряда, а также 4-импульса. Такая 
технология называется тагированием \textcolor{red}{ссылка на работу}. В 
результате получим, что в $X_c$ будет входить хотя бы один барион и кварки 
$u c d$, а также любые пары $q \bar{q}$. В итоге возможны следующие варианты 
$X_c$. Также важно понимать, что чем больше частиц содержит $X_c$, тем менее 
вероятно событие с такой комбинацией, так как новые частицы требуют 
дополнительных кварковых пар, создание которых требует больше энергии. 
Кроме того, при добавлении новых частиц время работы программы увеличивается 
экспоненциально, так как сложность алгоритма $\mathcal{O}(\prod_n N_n)$ 
(где $N_n$ количестов задетектированных частиц типа $n$ в событии).

Я рассмотрю $X_c \to \Lambda^{tag}_c; \Lambda^{tag}_c \pi^- \pi^-; \Lambda^{tag}_c \pi^+ \pi^- \pi^+ \pi^-; D^0 p; D^+ p \pi^-; D*^0 p; D*^+ p \pi^- $, чтобы отличать $\Lambda_c$ котрую мы такгируем от той что является продуктом $X_c$, вторую обозначаю как $\Lambda^{tag}_c$.
Каналы распада прочих частиц, так как для этой отальных частиц нам уже извесины бранчинговые соотношения, будут использованы самые эффективные из них согласно \cite{PDGTablesBar} для барионов и \cite{PDGTablesMes} для мезонов.
\begin{figure}[h]
    \centering
    \begin{tabular}{c|c}
        Particle & Channels \\ \hline
        $D^0$ & $K^- \pi^+; K^- \pi^+ \pi^+ \pi^-; K^- K^+; K^0_s \pi^+ \pi^+; K^0_s \pi^0; K^+ K^- K_s^0$ \\
        $D^+$ & $K^- \pi^+ \pi^+; K^0_s; K^0_s \pi^+ \pi^+ \pi^-; K^+ K^- \pi^+$ \\
        $\Lambda^{tag}_c$ & $pK^-\pi^+; \Lambda^0 \pi^+; \Lambda^0 \pi^+ \pi^0; p K_s^0 \pi^0$ \\
        $D*^+$ & $D^0 \pi^0; D^0 \gamma$ \\
        $D*^+$ & $D^+ \pi^0; D^0 \pi^+$ \\
        $\pi^0$ & $\gamma \gamma$ \\
        $K_s^0$ & $\pi^+ \pi^-$
    \end{tabular}
    \label{fig:part_channels}
\end{figure}

\subsection{Критерии отбора}

В данном разделе изложены критерии отбора, принятые на основании работы \cite*{BelleDetector2002}. Все аспекты, не рассмотренные в указанной работе, будут обсуждены отдельно.

\newdot На все фотоны, используемые при реконструкции событий, накладывается ограничение $E_\gamma > 50 \text{ MeV}$, поскольку фотоны с меньшей энергией трудно отличимы от индуцированных токов, что может привести к ошибочной интерпретации их как сигналы фотонов.

\newdot Идентификация частиц (PID):
Для идентификации частиц используются данные с детекторов CDC, ACC и TOF, которые позволяют вычислить функции правдоподобия $L_\pi, L_K, L_p$, соответствующие гипотезам пиона, каона и протона, соответственно. Для различения гипотез частиц a и b применяется условие на отношение:

$$\mathfrak{L}_{a/b} = \frac{L_a}{L_a + L_b}$$

\begin{figure}[h]
    \centering
    \begin{tabular}{c|c}
        $p$   & $\mathfrak{L}_{p/K} < 0.6; \mathfrak{L}_{p/\pi} > 0.6$ \\
        $K^0$   & $\mathfrak{L}_{p/K} < 0.4; \mathfrak{L}_{K/\pi} > 0.6$ \\
        $\pi^{\pm}$ & все треки, не прошедшие идентификацию по вышеуказанным критериям
    \end{tabular}
\end{figure}

\newdot $K_s^0$-мезоны реконструируются по распаду $K_s^0 \to \pi^+ \pi^-$ из кандидатов, отобранных с помощью стандартного инструмента V0finder и собранных в таблице MdstVee2. Критерии отбора следующие:

$$
\left| M_{K_s^0} - M^{real}_{K_s^0} \right| < 30 \text{ MeV}; \ \rho_{K_s^0} > 1 \text{ мм}; \ z_{K_s^0} > 1 \text{ см}; \ \cos \theta_{K_s^0} > 0.99
$$

где $M^{real}_{K_s^0} = 497.611 \text{ MeV}$, $M_{K_s^0}$ — инвариантная масса пионов ($\pi^+ \pi^-$), собранных в $K_s^0$-мезон, $z_{K_s^0}$ и $\rho_{K_s^0}$ — цилиндрические координаты реконструированной вершины распада $K_s^0$-мезона в лабораторной системе отсчёта, а $\cos \theta_{K_s^0}$ — азимутальный угол между импульсом $K_s^0$ и направлением на его вершину распада.

Независимо от работы \cite{BelleDetector2002}, среди пар $\pi^+ \pi^-$, 
идентифицированных как кандидаты на дочерние продукты распада $K_s^0$-мезона, 
можно использовать знание о том, что импульсы продуктов распада должны исходить 
из вершины распада. Это позволяет откорректировать измеренные импульсы с учётом 
погрешностей, чтобы они соответствовали данной гипотезе (в дальнейшем это будет 
называться "фит в вершину"). Аналогично, на основании инвариантной массы, 
известной для $K_s^0$, можно корректировать величины импульсов дочерних частиц 
так, чтобы $M_{K_s^0}$ совпадала с $M^{real}_{K_s^0}$. Этот метод будет 
называться "фит в массу". Были использованы алгоритмы для фита в вершину и 
в массу принятые в коллаборации KEK, и описанные в \cite{Krohn2021}.

\newdot $\pi^0$ -мезоны восстанавливались в распаде на два фотона, которые в
свою очередь реконструировались по кластерам энерговыделения в ECL.
Критерии отбора:
$$
\abs{M_{\pi^0} - M_{\pi^0}^{real}} < 15 MeV 
$$
После отбора стандартно были установлены погрешности для импульсов фотонов и выполнены фиты в вершину и массу.

\newdot Отбор $D$ -мезонов:

\begin{figure}[h]
    \centering
    \begin{tabular}{c|c}
        $D^0$ & $\abs{M_{D^0} - M^{real}_{D^0}} < 15 MeV$ \\
        $D^\pm$ & $\abs{M_{D^\pm} - M^{real}_{D^\pm}} < 15 MeV$ \\
        $D*^{\pm}$ & $\abs{M_{D*^\pm} - M^{real}_{D*^\pm}} < 3 MeV$ \\
        $D*^0$ & $\abs{M_{D*^0} - M^{real}_{D*^0}} < 3 MeV$ \\
    \end{tabular}
\end{figure}

Где $M^{real}_{D^\pm} = 1864.83 MeV; M^{real}_{D^0} = 1869.65 MeV; 
M^{real}_{D*^\pm} = 2010.26;  M^{real}_{D*^0} = 2006.85 MeV$. 




